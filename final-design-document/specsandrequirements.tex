\section{Specifications and Requirements}

\subsection{DAW Integrated Keyboard}

This project aims to create a complete MIDI controller with an integrated and complete
Digital Audio Workstation(DAW) in the system. The DAW will be able to interact with the
user via the keyboard, which acts as the MIDI controller, and can export a completed MIDI
file to a third party computer. For stability purposes, a more refined material will
likely be acquired to use in our final design while printed products will be utilized for
prototyping and visualization purposes.

This printed product will need to be created in sections to allow for printing from a
smaller 3D printing platform, but such a feature is easily designed around and is a common
tactic for printing much larger projects than the printer is normally capable of, but
sizable enough to house the necessary electrical components for our project.

The DAW itself will be hosted on an on board computer and be utilized to enact and execute
the various other features that are mentioned in this section and the document as a whole.
These features are subject to change and the exact method of execution will also likely
change to best accommodate our update in available information. Upon my writing this, we
have already been informed of a more preferential AI software to utilize for our system as
well as a handful of other AI that already exist that could be utilized in our project.
There is no telling where this project will lead us. Only time will tell.

\subsection{Autofill}

The Autofill feature should be capable of receiving a MIDI sequence as input, and producing
a new MIDI sequence which concatenates AI-generated MIDI notes to the end of the input sequence.
The AI should produce notes from the same scale as the input at least 90% of the time. It should
also produce notes within a perfect fifth interval (seven semitones) of the preceding note 90% of
the time. This will maintain the conjunctness of the melody - further discussed in \textbf{\ref{sec:theory} \nameref{sec:theory}}.

As a stretch goal, this feature should be able to implement influence's from the player's own style
by occasionally imitating melodic and/or rhythmic patterns from the input.

\subsection{Battery Life}

The device will be able to remain operational for six hours disconnected to any given
power supply. This will allow the user to change location and find a place of inspiration
before composing a musical piece. This will also encourage us to prioritize processor
usage to ensure that no overuse of resources or space occurs on our internal system. The
battery life will likely come in the form of internal packs that can be charged and
discharged depending on the requirements of the user as well as the power modes for the
internal computer.

Energy efficiency will be critical as well considering much of the pollution that occurs
in the world is due to runoffs from lithium mining and such materials in the soil have a
harsh environmental impact so limiting our required batteries while maximizing our battery
life is and absolute must in the situation.

\subsection{Energy Consumption}

The device will consume no more than 12 watts in a given moment. This will allow for an
extended battery life to longer times and mitigate the environmental impact of the MIDI
controller. This will also force us to limit the energy consumption via efficiency methods
and further allow our previous goal of battery life to come to fruition.

The consumption of energy also carries an environmental weight so limiting the amount of
energy our construction produces will help prevent the waste of electrical resources or
battery integrity.

\subsection{Processing Time}

The time it takes to compute a given function will take no longer than 5 seconds, limiting
user frustration upon perceived lag. This will also force an efficient solution to the
Energy consumption problem as the longer the device uses the main computing unit the more
power it will consume. Such computing problems happen on a near consistent basis and
utilizing optimization will be an essential technique in any field of computer science or
engineering.

This requirement will also force us to think creative when utilizing and implementing the
AI as these programs are notoriously tricky and time consuming, both to train and to
implement properly.
