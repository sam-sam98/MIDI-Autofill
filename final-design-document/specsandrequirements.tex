\section{Specifications and Requirements}

\subsection{DAW Integrated Keyboard}

This project aims to create a complete MIDI controller with an integrated and complete
Digital Audio Workstation(DAW) in the system. The DAW will be able to interact with the
user via the keyboard, which acts as the MIDI controller, and can export a completed MIDI
file to a third party computer. For stability purposes, a more refined material will
likely be acquired to use in our final design while printed products will be utilized for
prototyping and visualization purposes.

This printed product will need to be created in sections to allow for printing from a
smaller 3D printing platform, but such a feature is easily designed around and is a common
tactic for printing much larger projects than the printer is normally capable of, but
sizable enough to house the necessary electrical components for our project.

The DAW itself will be hosted on an on board computer and be utilized to enact and execute
the various other features that are mentioned in this section and the document as a whole.
These features are subject to change and the exact method of execution will also likely
change to best accommodate our update in available information. Upon my writing this, we
have already been informed of a more preferential AI software to utilize for our system as
well as a handful of other AI that already exist that could be utilized in our project.
There is no telling where this project will lead us. Only time will tell.

\subsection{Quantization}

The DAW will be able to Quantize a user recording to best fit the perceived tempo of the
recording being played. This is a standard feature for most all MIDI integrated keyboards
and doe not mark a particularly substantial point in this project's history, but such a
feature would be useful to aspiring artists so it would be necessary to include. Users
will also be able to adjust said recording to find the perfect location for each note on
the score and adjust said score to their liking should the quantization not behave as
intended for the artist.

This quantization feature will aid in the AI's interpretation and synthesis of new
melodies and the correction of existing notes as this precise method of music synthesis
limits the data needed to perfectly contain a given piece or melody. If each piece were
even a little larger it may cascade into a much more extensive or strenuous process to
contain, train, and utilize the AI to its desired function.

This feature will also extend to any MIDI file that is given as an incorrect timing
format, ensuring that, no matter the corruption, a valid MIDI file can be created upon
synthesization and output of the generated MIDI file. This will be essential in ensuring
no faulty or otherwise corrupted MIDI files are created or generated.

\subsection{Auto Fill}

The device will be able to fill in a melody to help the user create a unique piece of
music that falls into a specific genre selected. These Genres can be further trained and
improved through third party computation as well as customization of AI utilizing a
tailored list of songs as training files. The autofill feature is the main focus of the
project as well as the autocomplete feature that allows for less experiences musicians to
create melodies similar to their favorite artist or genre.

This feature will be designed to adapt to many genres as well as artists and users should
they decide to create and train a new AI structure. The Autofill will gravitate towards a
set limit beyond the initial melody provided and attempt to quantize it succinctly into a
sensible structure to ensure this goal. Initial plans to parse the MIDI into a .PNG then
running it through a visual algorithm have been discouraged as such a method would be
ineffective and clunky for our desired goal.

The Auto-fill can also be tweaked and adjusted to the user's preference should the
resulting melody prove unsatisfactory. This tweaking and adjustments follow the same
principles and intention of teaching the user and aiding in their goal of creating a
unique or otherwise new melody. Naturally, it would eventually become difficult if not
impossible to accomplish such a goal but that is far in the future and does not seem like
a prudent issue at the moment. The feature also does not include linguistic additions so
there is still much for the user to add of their own volition.

\subsection{Auto Correct}

The device will be able to offer suggestions on a given piece to improve it to better fit
the genre selected. These will appear as helpful hints on the screen, much like an
autocorrect software would for a given piece of text. This will be accomplished and
improved upon using AI and other music writing techniques discussed in other sections of
this document. The autocorrect features can also be adjusted based off which genre or
artist was given in the training data of the base AI. The same AI would be utilized in
either instance of correction or completion, given the circumstances.

These corrections will also have a list of priorities and justifications to help improve
the user's trust in the system as well as allow them to determine if the correction is
desirable or if they wish to discard the suggestion. This feature is, along with the
Auto-fill feature, an essential requirement for our project and the accomplishment of this
goal is our number one priority for constructing and testing various possible method for
this Keyboard.

\subsection{Education Mode}

The device will be able to help in the teaching of newer users or musicians to master the
art of composition and playing melodies. This is another pivotal goal in our
project as helping newer artists become more experienced with the techniques of
professionals is an essential means of creating artists. This will likely take the form of
a series of practice pieces that allow the user to get comfortable with timing and key
presses that are essential in playing the Keyboard.

Practice pieces can also be created from almost any MIDI file and given a series of
limitations to allow for the user to practice said piece. This would likely take the form
of automatic pauses until the user presses the correct chord with a visual indication of
what kind is being waited on in the DAW display. This presence of this may change as the
workload is determined and the time requirements is found for the AI creation and circuit
construction for the internals of the MIDI keyboard.

\subsection{Battery Life}

The device will be able to remain operational for six hours disconnected to any given
power supply. This will allow the user to change location and find a place of inspiration
before composing a musical piece. This will also encourage us to prioritize processor
usage to ensure that no overuse of resources or space occurs on our internal system. The
battery life will likely come in the form of internal packs that can be charged and
discharged depending on the requirements of the user as well as the power modes for the
internal computer.

Energy efficiency will be critical as well considering much of the pollution that occurs
in the world is due to runoffs from lithium mining and such materials in the soil have a
harsh environmental impact so limiting our required batteries while maximizing our battery
life is and absolute must in the situation.

\subsection{Energy Consumption}

The device will consume no more than 12 watts in a given moment. This will allow for an
extended battery life to longer times and mitigate the environmental impact of the MIDI
controller. This will also force us to limit the energy consumption via efficiency methods
and further allow our previous goal of battery life to come to fruition.

The consumption of energy also carries an environmental weight so limiting the amount of
energy our construction produces will help prevent the waste of electrical resources or
battery integrity.

\subsection{Processing Time}

The time it takes to compute a given function will take no longer than 5 seconds, limiting
user frustration upon perceived lag. This will also force an efficient solution to the
Energy consumption problem as the longer the device uses the main computing unit the more
power it will consume. Such computing problems happen on a near consistent basis and
utilizing optimization will be an essential technique in any field of computer science or
engineering.

This requirement will also force us to think creative when utilizing and implementing the
AI as these programs are notoriously tricky and time consuming, both to train and to
implement properly.
